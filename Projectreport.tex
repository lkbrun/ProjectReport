\documentclass[11pt,english,a4paper]{report}

\usepackage[toc,xindy]{glossaries} 

\title{INF226 - Project Report}
\date{\today}
\author{Lasse K. Brun and Håvard R. Olsen}

\newglossaryentry{endeffector}		{ name=End-effector,		description={is the device at the end of a rotics arm, designed to interact with the environment }}

\newacronym{dhis}{DHIS 2}{District Health Information System 2}
\newacronym{zap}{ZAP}{OWASP Zed Attack Proxy}
\newacronym{for}{FOR}{HP Fortify}
\newacronym{gis}{GIS}{Geographic Information System}

\makeglossaries

\begin{document}

\maketitle

\tableofcontents
\newpage

% Glossary
\printglossaries
\newpage


\chapter{Introduction}
\textit{This chapter will provide you with some background material for the report, why we are writing it and what we hope to accomplish.}

\section{Motivation}
\paragraph{}
In today's software world, security is more important than ever. New and refined attack methods keep emerging, making creation of secure software applications a full time job. This has caused production of new, innovative tools and frameworks to handle analysis of software. The tools and frameworks help software developers to find security breaches within their application. Throughout this report will the two most common tools be introduced, and tested on an actual codebase.

\paragraph{}
The codebase we will be using is gls{dhis}. \gls{dhis} is a flexible, web-based open-source information system. The system has visualization features as \gls{gis}, charts and pivot tables. 

\paragraph{}
The system is very widespread being the preferred health management information system in 46 countries across five continents. The health system helps governments and health organizations to manage their operations more efficiently, monitor processes and improve communication. The system is portable and can capture data on any type of device, including desktops, laptops, tablets and smartphones. 

\section{Goal}
\paragraph{}
The goal of this report is to expose security breaches within the provided codebase(\gls{dhis}). This will give us a better understanding of how the analysing tools work and how to use them. Thereafter must an evaluation of the result given by the tools be done. The evaluation will result in a conclusion, centred around a solution to the security flaws. By giving this report to the company responsible for the codebase, will hopefully give them an understanding of the security holes in their application and a solution to how they can solve them.

\chapter{Installation of tools and code base}
\textit{Here we talk about the the two tools we are using, and the codebase.}
\section{OWASP Zed Attack Proxy}
\paragraph{•}
\gls{zap} is a integrated penetration testing tool used for finding flaws and vulnerabilities in web applications. \gls{zap} is an excellent tool for both beginners and security experts as it provides an easy to use graphical interface as well as a more in-depth command line.
\paragraph{•}
In this section we will cover some of the features that \gls{zap} provides, and see how we can benefit from them.
\subsection{Features}
\subsubsection{Intercepting proxy}
(You can configure your browser to proxy through zap) So it can see all requests and responses. You can also intercepts and change them.
\subsubsection{Passive and Active scanner}
The passive scanner: Examines the requests and respo, but can detect problems just based on that. Safe to use on any site as it only listens to traffic, and doesn't do any attacks
The active scanner: It performs many different attacks on the site.

\subsubsection{Spider}
The spider crawls the application for example to find pages that you've missed or is hidden.
\subsubsection{Report Generation}
Gives detailed information about problems, how to learn more about that class of problems and also how to solve them.

\section{HP Fortify}
\subsection{Installation}
\subsection{Features and benefits}
\section{Code base}
\subsection{Installation}
\subsection{Features and benefits}



\chapter{Static Analysis and Penetration Testing Results}
\chapter{Analysis, Evaluation and Recommendations}

\end{document}

