\documentclass[11pt,english,a4paper]{report}

\title{INF226 - Project Report}
\date{\today}
\author{Lasse K. Brun and Håvard R. Olsen}


\begin{document}

\maketitle


\chapter{Introduction}
\textit{Here we are talking about the background for the report, why we are writing it and what we hope to accomplice.}

\section{Motivation}
In todays software world, security is more important than ever. Every week there is a new threat that needs to be handled. Because of this many tools and frameworks for analysing software have been created. These tools help software developers or security experts find weaknesses or flaws in their applications. We will in this report introduce two of the best tools out there, as well as testing them out on a actual codebase. 
\paragraph{•}
The codebase we will be using is dhis2(Short explanation about dhis2). By analysing this codebase we hope that we can uncover potential security holes in the application which we than can come up with solutions for.

\section{Goal}
\chapter{Installation of tools and code base}
\textit{Here we talk about the the two tools we are using, and the codebase.}
\section{OWASP Zed Attack Proxy}
\paragraph{•}
\gls{zap} is a integrated penetration testing tool used for finding flaws and vulnerabilities in web applications. \gls{zap} is an excellent tool for both beginners and security experts as it provides an easy to use graphical interface as well as a more in-depth command line.
\paragraph{•}
In this section we will cover some of the features that \gls{zap} provides, and see how we can benefit from them.
\subsection{Features}
\subsubsection{Intercepting proxy}
(You can configure your browser to proxy through zap) So it can see all requests and responses. You can also intercepts and change them.
\subsubsection{Passive and Active scanner}
The passive scanner: Examines the requests and respo, but can detect problems just based on that. Safe to use on any site as it only listens to traffic, and doesn't do any attacks
The active scanner: It performs many different attacks on the site.

\subsubsection{Spider}
The spider crawls the application for example to find pages that you've missed or is hidden.
\subsubsection{Report Generation}
Gives detailed information about problems, how to learn more about that class of problems and also how to solve them.

\section{HP Fortify}
\subsection{Installation}
\subsection{Features and benefits}
\section{Code base}
\subsection{Installation}
\subsection{Features and benefits}

\chapter{Static Analysis and Penetration Testing Results}
\chapter{Analysis, Evaluation and Recommendations}

\end{document}

